\documentclass[fullpage]{article}

	\addtolength{\oddsidemargin}{-.875in}
	\addtolength{\evensidemargin}{-.875in}
	\addtolength{\textwidth}{1.75in}

	\addtolength{\topmargin}{-.875in}
	\addtolength{\textheight}{1.75in}


\usepackage[utf8]{inputenc}
\usepackage[english]{babel}
\usepackage{amsthm}
\usepackage{placeins}

\makeatletter
\def\th@plain{%
  \thm@notefont{}% same as heading font
  \itshape % body font
}
\def\th@definition{%
  \thm@notefont{}% same as heading font
  \normalfont % body font
}
\makeatother

\usepackage{hyperref}
\usepackage{amsfonts}
\usepackage{amssymb}
\usepackage{mdframed}
\usepackage{extarrows}
\usepackage{caption}
\usepackage{xspace}
\usepackage{graphicx}
\usepackage{fixltx2e}
\usepackage[n,advantage,operators,sets,adversary,landau,probability,notions,logic,ff, mm, primitives,events,complexity,asymptotics,keys]{cryptocode}
\captionsetup[figure]{labelfont={bf},name={Fig.},labelsep=period}

\newtheorem{claim}{Claim}[section]
\newtheorem{conjecture}{Conjecture}[section]
\newtheorem{lemma}{Lemma}[section]
\newtheorem{theorem}{Theorem}[section]
\theoremstyle{definition}
\newtheorem{definition}{Definition}[section]
\newtheorem{remark}{Remark}

\def\SPSB#1#2{\rlap{\textsuperscript{#1}}\SB{#2}}
\def\SP#1{\textsuperscript{#1}}
\def\SB#1{\textsubscript{#1}}

\newcommand{\EUFCMA}{\textsf{EUF-CMA}\xspace}
\newcommand{\EUFCMAVES}{\textsf{EUF-CMA}[\textsf{VES}]\xspace}
\newcommand{\EUFCMAVESot}{\textsf{EUF-CMA}[\textsf{VES}\textsubscript{ot}]\xspace}
\newcommand\getsdollar{\mathrel{{\leftarrow}\vcenter{\hbox{\tiny\rmfamily\upshape\$}}}}
\newcommand{\SE}{SE}
\newcommand{\otVES}{\textsf{VES}\textsubscript{ot}}
\newcommand{\OPCHECKMULTISIG}{\texttt{OP\_CHECKMULTISIG}\xspace}
\newcommand{\Dec}{\textsf{Dec}}


\newcommand{\Enc}{\textsf{Enc}}
\newcommand{\EncGen}{\textsf{EncGen}}
\newcommand{\EncSign}{\textsf{EncSign}}
\newcommand{\EncVer}{\textsf{EncVrfy}}

\newcommand{\DecSig}{\textsf{DecSig}}
\newcommand{\KeyGen}{\textsf{Gen}}
\newcommand{\Sign}{\textsf{Sign}}
\newcommand{\Verify}{\textsf{Vrfy}}
\newcommand{\Rec}{\textsf{Rec}}
\newcommand{\RecKey}{\textsf{RecKey}}
\newcommand{\rec}{\delta}
\newcommand{\VESALG}{\EncGen, \EncSign, \EncVer, \DecSig}
\newcommand{\SIGNALG}{\KeyGen, \Sign, \Verify}
\newcommand{\ENCALG}{\EncGen, \Enc, \Dec}

\newcommand{\skSign}{sk_S}

\newcommand{\pkSign}{pk_S}
\newcommand{\kSign}{(\skSign,\pkSign)}
\newcommand{\skEnc}{sk_E}
\newcommand{\pkEnc}{pk_E}
\newcommand{\kEnc}{(\skEnc, \pkEnc)}
\newcommand{\F}{\mathcal{F}}
\newcommand{\hatsigma}{\hat{\sigma}}
\newcommand{\hatSigma}{\hat{\Sigma}}
\newcommand{\FullEncVer}{\EncVer(\pkSign, \pkEnc, m, \hatsigma)}
\newcommand{\FullVer}{\Verify(\pkSign,m,\sigma)}
\newcommand{\R}{\mathcal{R}}
\newcommand{\RDL}{\mathcal{R}_{\DLOG}}
\newcommand{\hatR}{\hat{\mathcal{R}}}
\newcommand{\hardproblem}{\mathcal{H}}
\newcommand{\G}{\mathbb{G}}
\newcommand{\DLOG}{\textsf{DL}\xspace}
\newcommand{\Setup}{\textsf{Setup}}


\begin{document}

\title{One-Time Verifiably Encrypted Signatures \\ A.K.A. Adaptor Signatures}
\author{Lloyd Fournier\footnote{\texttt{lloyd.fourn@gmail.com}}}
\date{October 2019}

\maketitle

\begin{abstract}
  On Bitcoin-like ledgers, smart contract functionality can be realised without using the ledger's native smart contract language through the ``adaptor signature'' technique\cite{poelstra-adaptor}. An adaptor signature is a kind of partial signature that, if completed, will reveal valuable information to the signer. In this work, we conduct the first formal analysis of the adaptor signature as an isolated primitive. We find that it offers similar functionality to the already well established concept of \emph{verifiably encrypted signatures} (VES) with one notable difference: the decryption key can be recovered from the ciphertext and the decrypted signature. To capture this property, we formally introduce the notion of \emph{one-time verifiably encrypted signatures}. To properly define their security in the Bitcoin layer-2 setting we revisit the original VES definitions and modify them to remove the ingrained assumption of a trusted third party.

  After the extending our VES definitions to the one-time VES, we attempt to prove that the existing Schnorr and ECDSA adaptor signature schemes satisfy them. In the case of Schnorr, we succeed unconditionally but our definitions expose a (non-fatal) flaw in the ECDSA scheme. Nevertheless, we show how to use the ECDSA scheme to realise functionality in Bitcoin that was previously thought to be out of reach without Schnorr signatures or complex two-party ECDSA protocols.
\end{abstract}


% CDH problem?

% VERIFIABLY ENCRYPTED SIGNATURES

%%% Local Variables:
%%% mode: latex
%%% TeX-master: "main"
%%% End:
\section{Verifiably Encrypted Signatures Without a Trusted Third Party}
\label{VES-section}

Verifiably encrypted signatures (VES) were introduced by Boneh, Gentry, Lynn and Shacham (BGLS) \cite{Boneh:2003:AVE:1766171.1766207} in 2003 for the BLS signature scheme\cite{Boneh:2001:SSW:647097.717005}. A VES scheme lets a signer create a signature encryption that can be non-interactively verified. Just by knowing the message and public signing key, a verifier can tell that a ciphertext contains a valid signature encrypted by a particular encryption key. This separates it from the earlier notion of \emph{signcryption}\cite{signcryption-book} where the message is also encrypted and the verification is often interactive.

This idea is somewhat unintuitive. If I can verify that the signer has indeed signed the message then what's the point of decrypting the ciphertext? A VES is only useful in settings where the signature itself is what has value rather than the fact that someone has signed it. In layer-2 protocols we have exactly this situation. Having a verifiably encrypted transaction signature is not enough to make the transaction valid --- it must first be decrypted and attached to the transaction. Skipping ahead a bit, with our new definitions we do not even require that a valid encrypted signature was created by the signer for it to be a secure VES scheme.

The original definitions of BGLS were made relative to a trusted third party, the \emph{adjudicator}. The original idea was that two parties could optimistically exchange signatures, by first exchanging verifiably encrypted signatures, then should one of them fail to provide their signature, the other could go to the adjudicator to have it decrypted. This is not appropriate for our setting. In most of the protocols described in Section~\ref{exisitng-protocols}, one of the possibly malicious parties generates the encryption key-pair. We take our first step towards a trusted party free definition by removing references to the ``adjudicator'' from the definition of VES:


\begin{definition}[Verifiably Encrypted Signature Scheme]
\label{VES}
 A verifiably encrypted signature scheme (VES) $\hatSigma$ is defined with an ordinary \emph{underlying} signature scheme $\Sigma := (\SIGNALG)$ and four additional algorithms:
    \begin{itemize}
        \item $\EncGen \rightsample \kEnc$: A probabilistic encryption key generation algorithm which outputs an encryption key $\pkEnc$ and a decryption key $\skEnc$. There should also exist an efficient predicate $\textsf{valid}\kEnc$ that returns 1 when $\kEnc$ is a valid key-pair.
        \item $\EncSign(\skSign, \pkEnc, m) \rightsample \hatsigma$: A possibly probabilistic encrypted signing algorithm, which on input of a secret signing key $\skSign$, a public encryption key $\pkEnc$ and a message $m$ outputs a ciphertext $\hatsigma$.
        \item $\EncVer(\pkSign, \pkEnc, m, \hatsigma) \rightarrow \bin$: A deterministic encrypted signature verification algorithm which on input of a public signing key $\pkSign$, a public encryption key $\pkEnc$, a message $m$ and a ciphertext $\hatsigma$ outputs 1 only if $\hatsigma$ is a valid encryption of a signature on $m$ for $\pkSign$ under $\pkEnc$.
        \item $\DecSig(\skEnc, \hatsigma) \rightarrow \sigma$: A (usually) deterministic signature decryption algorithm which on input of a decryption key $\skEnc$ and a valid ciphertext $\hatsigma$ under that encryption key  outputs a valid signature $\sigma$.
    \end{itemize}

Any coherent VES should satisfy a basic notion of completeness such that for all messages $m$, valid encryption and signing key pairs $\kEnc$ and $\kSign$ and for all coin tosses of $\EncSign$ the following always holds:
    \[ \hatsigma = \EncSign(\skSign, \pkEnc, m) \implies \EncVer(\pkSign, \pkEnc, m, \hatsigma) = 1 \land \Verify(\pkSign, m, \DecSig(\skEnc, \hatsigma)) = 1 \]
\end{definition}

The original work BGLS proposed three security properties: \emph{validity}, \emph{unforgeability}, and \emph{opacity}. To meet the requirements of our setting, we will restate validity without reference to a trusted party and replace unforgeability with a new \emph{existential unforgeability under chosen message attack} (\EUFCMAVES) property. We keep the original definition of opacity. Therefore our VES requirements are informally as follows:

\begin{itemize}
    \item \textbf{Validity:} It is infeasible to generate a valid looking ciphertext that does not yield a valid signature upon decryption.
    \item \EUFCMAVES: A VES ciphertext does not help an adversary forge signatures. In other words, an adversary cannot learn anything useful from a VES ciphertext other than the signature that is encrypted.
    \item \textbf{Opacity:} The encrypted signature cannot be extracted from the ciphertext without the decryption key.
\end{itemize}


We formally present our new definitions for validity and introduce \EUFCMAVES after summarizing previous work on improving VES definitions. We do not formally describe opacity as the original definitions are not applicable the one-time VES (see Section~\ref{otVES}).

\subsection{Previous Revisions of the VES Security Definitions}

Rückert et al.\ \cite{Ruckert:2009:SVE:1615384.1615387} noticed that \emph{key independent} schemes (which we call Sign then Encrypt (StE)) whose underlying signature schemes are \EUFCMA secure can be proved unforgeable generically.
This is remarkable as the original BGLS scheme, which is StE, had an involved proof for unforgeability that spanned several pages.
We use a similar idea to prove any StE scheme satisfies \EUFCMAVES\@. Hanser et al.\ \cite{VES-structure-preserving} also identified the importance of the $\EUFCMA$ security of the underlying signature scheme. They proved that the underlying signature scheme is unforgeable if the VES scheme is unforgeable based on a different property.

Calderon et al.\ \cite{calderon2014rethinking} discuss the original definitions in detail.
They show a pathological VES scheme which is secure according to the original definitions but can be constructed only using a signature scheme i.e.\ without any encryption.
For simplicity, our definitions exclude the pathological constructions but could be easily modified to account for them.

Shao \cite{SHAO20081961} addresses the assumption of trust in the adjudicator by providing a stronger definition of unforgeability which prevents forging a VES even if the adjudicator is corrupted.
Unfortunately, the original and highly practical VES scheme of BGLS does not meet this stronger requirement.
Our approach to removing trust in a third party is to simply replace the concept of unforgeability with \EUFCMAVES\@.
The ability of malicious parties to forge signature encryptions is not a concern in our setting as long as they cannot forge signatures (which is already ensured by the \EUFCMA security of the signature scheme).

\subsection{Validity}

Validity protects against an adversary who attempts to create a valid looking ciphertext that when decrypted will not yield a valid signature.
The original BGLS definition of validity was inadequate as they only guaranteed a valid signature upon decryption if the ciphertext was generated by the $\EncSign$ algorithm.
Rückert et al. \cite{Ruckert:2009:SVE:1615384.1615387} noticed this problem and introduced the additional property of \emph{extractability} which ensured the adversary could not find a malicious ciphertext against the trusted adjudicator's encryption key.

This definition of extractability is not appropriate for our setting as we have no trusted party.
In layer-2 protocols, the encryption key and the signing key may be generated by the same malicious party.
For example, imagine an adversary who maliciously generates a VES ciphertext on some transaction signature and attempts to sell the decryption key to someone who wishes to know the signature.
If they are successful they can get paid for the decryption key but even after obtaining the decryption key the buyer will be unable to get their desired signature.
Therefore, in our setting, the adversary must be free to choose the signing and encryption keys.
We choose to use the original name of validity to capture this idea as it ensures that the validity of a ciphertext carries through to the validity of the decrypted signature.

\begin{definition}[Validity]
 A VES scheme $\hatSigma$ is $(\tau, \epsilon)$-valid if $\prob{\textsf{VES-Validity}^{\adv}_{\hatSigma} = 1} \leq \epsilon$, for all algorithms $\adv$ running it at most time $\tau$.
\begin{center}
    \fbox{%
          \procedure{$\textsf{VES-Validity}^{\adv}_{\hatSigma}$}{%
          (\pkSign,(\skEnc, \pkEnc),m, \hatsigma) \sample \adv \\
          \sigma \gets \DecSig(\skEnc, \hatsigma) \\
          \pcreturn \pcalgostyle{valid}(\skEnc, \pkEnc) \implies \EncVer(\pkSign, \pkEnc, m, \hatsigma) = \Verify(\pkSign, m, \sigma)
    }
    }
\end{center}
\end{definition}
\subsection{\EUFCMAVES}

Our goal in this section is to formally capture the requirement that nothing can be learned from a VES ciphertext except the signature encrypted within.
We start with the typical $\EUFCMA$ security definition for signature schemes which ensures that from a signature, nothing can be learned about a signature on any other message under the same signing key.
The experiment tests this by giving the forger a signature oracle $S$ from which it can query signatures under the signing key on messages of its choosing.
For a secure scheme, no algorithm should exist that is able to forge signatures even with access to $S$.
By implication, the forger learns nothing useful from the signatures other than the signatures themselves.

To show that a VES ciphertext equally offers no extra information to the forger, we modify the experiment so the forger instead has access to a signature encryption oracle $E$.
The forger may query this oracle under any encryption key and message it wants and receives back a valid signature on the message encrypted under that key.
This allows, for example, the forger to request a ciphertext where the encryption key is a functions of the signing key they are trying to forge against in the hope that this will leak something about the signing key.
This is crucial because as we will see later, due to a problem in our ECDSA one-time VES scheme, the forger can do exactly this to speed up an attempt to recover the secret signing key.
Note that we provide the encryption oracle $E$ instead of $S$, rather than in addition to, because we assume $S$ can be simulated with $E$ by simply requesting an encryption for which the forger knows the decryption key.
This assumption holds as long as decrypted signatures are indistinguishable from ordinary signatures, which is true except for some pathological constructions\cite{calderon2014rethinking}.
We call denote this modified experiment $\EUFCMAVES$ and define it as follows.

\begin{definition}[\EUFCMAVES]
  A VES scheme $\hatSigma$ is $(\epsilon, \tau, Q_E)$-\EUFCMAVES secure if $\prob{ \EUFCMAVES^{\F}_{\hat{\Sigma}} = 1} \leq \epsilon$ for all forgers $\F$ making at most $Q_E$ signature encryption queries and running in at most time $\tau$.
\begin{center}
\fbox{
    \procedure{$\EUFCMAVES^{\F}_{\hat{\Sigma}}$}{
        Q := \emptyset \\
        (\skSign,\pkSign) \sample \KeyGen \\
        (m^*, \sigma) \sample \F^{E}(\pkSign) \\
        \pcreturn \Verify(\pkSign, m^*, \sigma) \land m^* \notin Q \\
    }
    \pchspace
    \procedure{Oracle $E(\pkEnc, m)$}{
        Q := Q \cup \{m\}\\
        \pcreturn \EncSign(\skSign,\pkEnc,m) \\
    }
}
\end{center}
\end{definition}

$\EUFCMAVES$ effectively replaces the original definition of \emph{unforgeability} so we now discuss and prove the relationship between the two.
The original unforgeability property referred to signature encryptions being unforgeable under the trusted adjudicator's encryption key and is therefore inadequate for our setting.
\EUFCMAVES says nothing about the unforgeability of signature encryptions.
In fact, an adversary who can produce valid VES ciphertexts without the secret signing key is perfectly compatible. % TODO: show algorithms for doing this
Of course, they will never be able to forge a VES ciphertext under a particular encryption key.
If they could do that, then they could trivially forge an encrypted signature under a key for which they know the decryption key and decrypt it.
The original definitions seem to have missed this intuition: it is not the security of the VES scheme that prevents there being a successful forger of signature encryptions under a particular key, the \EUFCMA security of the underlying signature scheme already ensures that no such algorithm can exist.
After recalling the original BGLS definition of unforgeability, we use this intuition to prove that any \EUFCMAVES scheme is also unforgeable.

\begin{definition}[BGLS VES Unforgeability \cite{Boneh:2003:AVE:1766171.1766207}]
\label{original-unforgeability}

We say a VES scheme $\hatSigma$ is $(\tau, \epsilon, Q_E, Q_D)$-BGLS unforgeable if $\prob{\textsf{VES-Forge} = 1} < \epsilon$ for all algorithms $\F$ running in time $\tau$ making $Q_E$ and $Q_D$ signature encryption and decryption oracle queries respectively.
Note that unlike in the $\EUFCMAVES$ experiment the oracles in $\textsf{VES-Forge}$ only provide signature encryptions and decryption on a static key chosen by the experiment (this represents the trusted adjudicator's key).

 \begin{center}
 \fbox{
 \begin{pchstack}
       \procedure{$\textsf{VES-Forge}^{\F}_{\hatSigma}$}{
        Q \gets \emptyset \\
        \kSign \sample \KeyGen \\
        \kEnc \sample \EncGen \\
        (m^*, \hatsigma^*) \sample \F^{\tilde{E},\tilde{D}}(\pkSign,\pkEnc) \\
        \pcreturn \EncVer(\pkSign, \pkEnc, m^*, \hatsigma^*) \land m^* \notin Q \\
     }
     \pchspace
     \procedure{Oracle $\tilde{E}(m)$}{
        Q := Q \cup \{m\} \\
        \hatsigma \sample \EncSign(\skSign,\pkEnc,m) \\
        \pcreturn \hatsigma
     }
     \pchspace
     \procedure{Oracle $\tilde{D}(\hatsigma)$}{
        \EncVer(\pkSign, \pkEnc, \hatsigma) \stackrel{?}{=} 1 \\
        \pcreturn \DecSig(\skEnc,\hatsigma) \\
     }
 \end{pchstack}
 }
 \end{center}


\end{definition}

\begin{theorem}[\EUFCMAVES + Validity $\implies$ BGLS Unforgeability]
If a VES scheme is $(\tau_f, \epsilon_f, Q_E)$-\EUFCMAVES secure and $(\tau_v, \epsilon_v)$-valid then it is $(\tau, \epsilon, Q_E, Q_D)$-BGLS unforgeable where $\epsilon \leq \epsilon_f + \epsilon_v$, $\tau = \tau_f - O(Q_E + Q_D)$ and $\tau_v \approx \tau$.
\end{theorem}

\begin{proof}

  We can bound the advantage of a BGLS forger by constructing an algorithm $\R_{\EUFCMAVES}$ which forge a signature in the $\EUFCMAVES$ experiment and an algorithm $\R_{\text{valid}}$  that outputs a valid ciphertext that does not decrypt to a valid signature.
  $\R_{\EUFCMAVES}$ is depicted below.
\begin{center}
\fbox{
\begin{pchstack}
\procedure{$\R^{E}_{\EUFCMAVES}(\pkSign)$}{%
    (\skEnc, \pkEnc) \sample \EncGen \\
    (m^*, \hatsigma^*) \sample \F^{\tilde{E},\tilde{D}}_{\textsf{VES-Forge}}(\pkSign, \pkEnc) \\
    \sigma^* \gets \DecSig(\skEnc, \hatsigma^*) \\
    \pcreturn (m^*, \sigma^*) \\
}
\pchspace
\procedure{Simulate $\tilde{E}(m)$}{
    \pcreturn E(\pkEnc,m) \\
}
\pchspace
\procedure{Simulate $\tilde{D}(\hatsigma)$}{
    \EncVer(\pkSign, \pkEnc, \hatsigma) \stackrel{?}{=} 1 \\
    \pcreturn \DecSig(\skEnc, \hatsigma)
}
\end{pchstack}
}
\end{center}


Clearly $\R$ perfectly simulates the view of $\F$ in the real experiment.
If $\F$ outputs a valid ciphertext that yields a valid signature upon decryption $\R_{\EUFCMAVES}$ successfully forges a signature in the $\EUFCMAVES$ experiment.
In the other case where $\F$ outputs a valid ciphertext that does not yield a valid signature we have broken validity.
Therefore we can bound $\epsilon$ (the success probability of $\F$) by $\epsilon_f + \epsilon_v$.
$\R$ takes only a constant factor longer than $\F$ to run in addition to answering its oracle queries so $\tau' = \tau - O(Q_E + Q_D)$.


\end{proof}

We now make the case for using \EUFCMAVES as the standard VES security definition even in the trusted party setting since it implies BGLS unforgeability and it is much easier to prove for the schemes that have been developed so far.
We can prove all existing schemes we are aware of \cite{Boneh:2003:AVE:1766171.1766207, Ruckert:2009:SVE:1615384.1615387, waters-ves, SHAO20081961, VES-structure-preserving} \EUFCMAVES secure just by the \emph{Sign then Encrypt} (StE) structure they all share.
That is, internally the $\EncSign$ algorithm first generates a normal signature using $\Sign$ and then encrypts the result. Formally:

\begin{definition}[Sign then Encrypt VES]
A \emph{Sign then Encrypt} (StE) VES scheme is defined with an ordinary \emph{underlying} signature scheme $\Sigma := (\SIGNALG)$ an \emph{associated} public key encryption scheme $\Pi = (\ENCALG)$ and a VES verification algorithm $\EncVer$ such that if we define $\DecSig := \Dec$ and $\EncSign(\skSign, \pkEnc, m) := \Enc(\pkEnc, \Sign(\pkSign, m))$ then $\hatSigma := (\EncGen, \EncSign, \EncVer, \DecSig)$ is a VES scheme according to Definition~\ref{VES}.
\end{definition}

It is easy to see that any valid StE scheme must be $\EUFCMAVES$ secure.
The encryption oracle $E$ the forger has access to in the $\EUFCMAVES$ experiment can be simulated just by encrypting the result of a query to the signature oracle $S$ that the \EUFCMA experiment provides.

\begin{theorem}[StE $\implies$ \EUFCMAVES]
Let $\hatSigma$ be a StE VES scheme,  $\Sigma$ be its underlying signature scheme and $\Pi = (\ENCALG)$ be its associated public key encryption scheme.
If there is reduction $\R$ from some hard problem $\hardproblem$ to the $\EUFCMA$ security of $\Sigma$, then there is a reduction $\hatR$ from $\hardproblem$ to the \EUFCMAVES security of $\hatSigma$.
\end{theorem}

\begin{proof}
  To solve $\hardproblem$, $\hatR$ runs $R$ internally, outputs whatever it outputs and lets it answer all oracle queries from the VES forger except for those directed to $E$.
  When $\hatR$ receives an $E$ oracle query $(m, \pkEnc)$ it makes a query $(m)$ to $S$ and receives $\sigma$.
  It then sets $\hatsigma \gets \Enc(\pkEnc, m)$ and returns $\hatsigma$ to the forger.
  Clearly, $\hatR$ is also a valid $\EUFCMAVES$ reduction from $\hardproblem$ to $\R$ is a valid $\EUFCMA$ reduction from $\hardproblem$.
\end{proof}

%%% Local Variables:
%%% mode: latex
%%% TeX-master: "main"
%%% End:
\section{One-Time Verifiably Encrypted Signatures}
\label{otVES}
We introduce the \emph{one-time verifiably encrypted signature scheme} to abstract the functionality of the ``adaptor signature''\cite{poelstra-adaptor}. A one-time VES is a VES with an additional property: given the ciphertext and the decrypted signature, the decryption key is easily recoverable.
Obviously, this property makes it useless for the optimistic fair exchange of signatures with a trusted party envisioned for ordinary VES schemes, as the trusted adjudicator would leak their decryption key after its first use.
Despite this, the property turns out to be incredibly useful to layer-2 protocol designers as it allows them to force the release of a secret key whenever a party broadcasts a decrypted transaction signature.
We define by extending the original VES definition:

\begin{definition}[One-Time Verifiably Encrypted Signatures]
A One-Time Verifiably Encrypted Signature scheme (one-time VES) is a Verifiably Encrypted Signature scheme (\SIGNALG, \VESALG) with the following additional algorithms:
\begin{itemize}
    \item $\RecKey(\pkEnc, \hatsigma) \rightarrow \delta$: A deterministic recovery key extraction algorithm which extracts a recovery key $\delta$ from the ciphertext $\hatsigma$ and the public encryption key $\pkEnc$.
    \item $\Rec(\sigma,\rec) \rightarrow \skEnc$: A deterministic decryption key recovery algorithm which when given a decrypted signature $\sigma$ and the recovery key $\rec$ associated with the original ciphertext, returns the secret decryption key $\skEnc$.
\end{itemize}

\end{definition}

Technically, $\Rec$ takes a ``recovery key'' $\delta$ rather than the ciphertext $\hatsigma$ but they should usually be thought of as equivalent. The distinction only really becomes necessary in Section~\ref{semi-scriptless} when we construct a simple two party encrypted signing protocol where one party is not able to output the whole ciphertext but is able to output the recovery key.

To be considered secure, we require a one-time VES scheme to have the \emph{completeness}, \emph{validity} and $\EUFCMAVES$ properties of an ordinary VES along with \emph{recoverability} which we define as follows.

\begin{definition}[Recoverability]

  A one-time VES scheme is $(\tau,\e)$-recoverable for a given security paramter $k$ if $\prob{\textsf{VES-Recover}^{\adv}_{\Setup}(k) = 1} \leq \e$ for all adversaries $\adv$ running in time at most $\tau$.
  The the probability is taken over the coin tosses of $\Setup$ in addition to those of $\adv$.

\begin{center}
    \fbox{%
          \procedure{$\textsf{VES-Recover}^{\adv}_{\Setup}$}{%
            \hatSigma \sample \Setup(1^k) \\
            (\pkSign,(\skEnc, \pkEnc), m,\hatsigma ) \sample \adv(\hatSigma) \\
            \delta \gets \hatSigma.\RecKey(\pkEnc, \hatsigma); \sigma \gets \hatSigma.\DecSig(\skEnc, \hatsigma) \\
            \pcreturn(\hatSigma.\textsf{valid}(\skEnc, \pkEnc) \land \hatSigma.\FullEncVer \land \hatSigma.\FullVer) \implies \hatSigma.\Rec(\sigma, \delta) = \skEnc
        }
    }

 \end{center}

\end{definition}

Note that we no longer have need of the \emph{opacity} property as defined for an ordinary VES scheme. Recall that the informal purpose of opacity is to ensure that an encrypted signature cannot be accessed without the decryption key. Recoverability makes this is trivial because if you are able to access the signature you can recover the decryption key anyway. In other words, obtaining a signature from a ciphertext without the decryption key is no easier than obtaining the decryption key just given the public encryption key (which is hopefully hard). Crucially, this also means that even the signer themselves cannot produce the signature from the ciphertext without the decryption key. This necessarily implies that one-time VES schemes only exist for probabilistic signature schemes where there is an exponential number of possible signatures under a public key for any message.

%%% Local Variables:
%%% mode: latex
%%% TeX-master: "main"
%%% End:
\section{Schnorr One-Time VES Scheme}
What we call the Schnorr one-time VES was introduced by Poelstra \cite{poelstra2017scriptless} which he termed an ``adaptor signature''\cite{poelstra-adaptor} with the encryption key being termed an ``auxiliary point''\cite{blind-tumbler} or sometimes just ``$T$''. A major benefit of our one-time VES concept is to be able to explain this useful idea with intuitive encryption/decryption semantics. It is typically described with a two-party signing protocol as this is where it is most useful. We describe the single singer scheme in Figure~\ref{fig:schnorr-ves} including its underlying Schnorr signature scheme. The Schnorr signature scheme was introduced by its namesake in \cite{Schnorr:1989:EIS:646754.705037} and our description resembles the key-prefixed scheme described in the Schnorr Bitcoin Improvement Proposal\cite{bip-schnorr} currently under consideration.

\begin{figure}[h]

    \centering
    \begin{mdframed}
    \begin{pchstack}[center]
    \procedure{$\KeyGen/\EncGen$}{
        x \sample \ZZ_q; X \gets g^{x} \\
        sk := (x,X); pk := X \\
        \pcreturn (sk,pk)
    }
    \pchspace
    \procedure{$\Sign(\skSign, m)$}{
        (x,X) := \skSign; \\
        r \sample \ZZ_q; R \gets g^r \\
        c := H(R || X || m) \\
        s \gets r + cx \\
        \pcreturn \sigma := (R,s) \\
    }

    \procedure{$\Verify(\pkSign,m,\sigma)$}{
        X := \pkSign; (R,s) := \sigma \\
        c := H(R || X || m) \\
        \pcreturn R = g^{s}X^{-c}
    }
    \pchspace
    \procedure{$\EncSign(\skSign,\pkEnc, m)$}{%
        (x,X) := \skSign; Y := \pkEnc  \\
        r \sample \ZZ_q; \hat{R} \gets{} g^r \\
        R \gets \hat{R}Y \\
         c := H(R || X || m) \\
        \hat{s} \gets r + cx \\
         \pcreturn \hat{\sigma} := (\hat{R},\hat{s}) \\
    }
    \end{pchstack}
    \begin{pchstack}[center]
    \procedure{$\EncVer(\pkSign,\pkEnc, m, \hatsigma)$}{%
        X := \pkSign; Y := \pkEnc \\
        (\hat{R},\hat{s}) := \hatsigma \\
        R \gets \hat{R}Y \\
        c := H(R || X || m) \\
        \pcreturn \hat{R} = g^{\hat{s}}X^{-c} \\
    }
    \pchspace
    \procedure{$\DecSig(\skEnc,\hatsigma)$}{
        (\hat{R},\hat{s}) := \hatsigma \\
        (y, Y) := \skEnc \\
        R \gets \hat{R}Y \\
        s \gets \hat{s} + y \\
        \pcreturn \sigma := (R,s)
    }
    \pchspace
    \procedure{$\RecKey(\pkEnc, \hatsigma)$}{
        (\hat{R},\hat{s}) := \hatsigma \\
        \pcreturn \rec := \hat{s} \\
    }
    \pchspace
    \procedure{$\Rec(\sigma, \rec)$}{
        (R,s) := \sigma \\
        \hat{s} := \rec \\
        y \gets  s - \hat{s} \\
        \pcreturn y
    }
    \end{pchstack}
    \end{mdframed}
    \caption{The algorithms of the Schnorr one-time VES scheme with a hash algorithm $H$}
    \label{fig:schnorr-ves}
\end{figure}

Since the $\EncSign$ algorithm is a simple tweak of the sign algorithm it is easy to get an intuition for the security of the scheme.
Unsurprisingly, we find that the scheme is unconditionally valid and recoverable and is \EUFCMAVES secure.
We now formally prove these properties.

\begin{lemma}
The Schnorr one-time VES is unconditionally valid and recoverable.
\end{lemma}

\begin{proof}
 Since $\EncVer(X,Y,m,(\hat{R},\hat{s})) = 1$ implies $ \hat{R} = g^{\hat{s}}X^{-c}$ where $c := H(\hat{R}Y || X || m)$, by the one-way homomorphism between $\ZZ_q$ and $\G$ we know that $\hatsigma$ encrypts some valid signature $(R,s) := (\hat{R}Y, \hat{s} + y)$ where $g^y = Y$ and is therefore valid because:
\[ \hat{R}Y = Yg^{\hat{s}}X^{-c} = g^{\hat{s} + y}X^{-c} \]
By the same token, the scheme is recoverable because $\Rec((\hat{R}Y, \hat{s} + y)), \hat{s})$ always returns the decryption key $y = s - \hat{s}$.
\end{proof}

\begin{theorem}
  If the Schnorr signature scheme is $(\tau,\epsilon,Q_H + Q_E)$-$\EUFKO$ secure in the random oracle model then the Schnorr one-time VES is $(\tau', \epsilon', Q_H, Q_E)$-$\EUFCMAVES$ secure where

  \[ \epsilon' \leq 4\epsilon + \frac{Q_HQ_E}{\abs{\G}}, \tau' \approx \tau \]

  and $Q_H,Q_E$ is the upper bound on the random oracle queries and signature encryption queries respectively in the $\EUFCMAVES$ experiment.
\end{theorem}

\begin{proof}
  The bound stated above identical the one presented by Kiltz et al. \cite{optimalsecschnorr} in Lemma 3.10 for $\EUFCMA$ except with the number of signature queries $Q_S$ to $S$ replaced with the number of signature encryption queries $Q_E$ to $E$.
  This is the correct bound since we can simulate $E$ in the same way as $S$ is usually simulated.

\begin{center}
  \fbox{%
        \procedure{Simulate $S(m)$}{
            s,c \sample \ZZ_q \\
            R \gets g^sX^{-c} \\
            \pcif H(R || X || m) = \bot \pcthen \\
             \t H(R || X || m) := c \\
             \t \pcreturn (R,s) \\
            \pcelse \textbf{abort} \\
        }
        \pchspace
        \procedure{Simulate $E(\pkEnc, m)$}{
             Y := \pkEnc \\
             \hat{s},c \sample \ZZ_q \\
             \hat{R} \gets g^{\hat{s}}X^{-c}; R \gets \hat{R}Y \\
             \pcif H(R || X || m) = \bot \pcthen \\
             \t H(R || X || m) := c \\
             \t \pcreturn (\hat{R},\hat{s}) \\
             \pcelse \textbf{abort} \\
        }
    }
  \end{center}

  Observe that the probability of the $E$ aborting is identical to that of $S$. Although $R$ is calculated differently it is distributed identically.
  $E$ runs in the same time as $S$ except for one extra group operation.
  Thus, assuming the bound from Kiltz et al.\ is correct, replacing $S$ with $E$ yields the same bound on the advantage of a $\EUFCMAVES$ forger except with $Q_S$ replaced with $Q_E$.
\end{proof}

%%% Local Variables:
%%% mode: latex
%%% TeX-master: "main"
%%% End:
\section{ECDSA One-Time VES Scheme}
\label{ecdsa-ot-ves-section}

\newcommand{\Pdleq}{\pcalgostyle{P}_{\textsf{DLEQ}}}
\newcommand{\Vdleq}{\pcalgostyle{V}_{\textsf{DLEQ}}}

In order circumvent the need for Schnorr signatures, which are not included in the Bitcoin protocol today, Moreno-Sanchez et al.\  developed an adaptor signature construction for the standard ECDSA signature scheme already used in Bitcoin\cite{ecdsa-scriptless-scripts}.
It was later applied to achieve a payment channel construction with better privacy in \cite{cryptoeprint:2018:472} and an efficient tumbler in \cite{cryptoeprint:2019:589}.
The scheme is built upon the two-party ECDSA protocol from \cite{Lindell2pECDSA}.
In Figure~\ref{fig:ecdsa-adaptor} we distill it into a single signer scheme which avoids all the complexities that come with two-party ECDSA protocols.

The construction works in a similar way to the Schnorr scheme: the public randomness $R$ is mutated independently of its private counterpart $r$ to include the encryption key $Y$.
This offsets the resulting signature by the same factor.
Unfortunately, due to the non-linear structure of ECDSA it needs a non-interactive zero knowledge proof of discrete logarithm equality so the verifier can confirm that $s$ is offset by the correct amount.
we denote the proof generation and verification algorithms as follows $(\Pdleq,\Vdleq)$.
Formally, When invoked as $\Pdleq((g, A),(h,B),w)$, it generates a proof of membership of the language:

\newcommand{\DLEQ}{\textsf{DLEQ}\xspace}
\[ L_{\DLEQ} = \{ (g, h, A, B) \in \G^4 \mid  \exists w \in \ZZ_q : A = g^w \land B = h^w \} \]

We instantiate this proof with the Fiat-Shamir transform applied to the Sigma protocol for the relation originally described in \cite{dleq-proof}.

\newcommand{\Rx}{R_{\mathtt{x}}}
\newcommand{\hatRx}{\hat{R}_\mathtt{x}}
\newcommand{\xcoord}{f}

\begin{figure}[h]
    \centering
    \begin{mdframed}
    \begin{pchstack}[center]
    \procedure{$\KeyGen/\EncGen$}{
        x \sample \ZZ_q; X \gets g^{x} \\
        sk := (x,X); pk := X \\
        \pcreturn (sk,pk)
    }
    \pchspace

    \procedure{$\Sign(\skSign, m)$}{
        (x,X) := \skSign; \\
        r \sample \ZZ_q; R \gets g^r \\
        \Rx \gets \xcoord(R) \\
        s \gets r^{-1}(H(m) + \Rx{}x)\\
        \pcreturn \sigma := (\Rx,s) \\
    }

    \procedure{$\Verify(\pkSign,m,\sigma)$}{
        X := \pkSign; (\Rx,s) := \sigma \\
        R' \gets (g^{H(m)}X^{\Rx})^{s^{-1}} \\
        \pcreturn \xcoord(R') = \Rx \\
    }
    \pchspace
    \procedure{$\EncSign(\skSign,\pkEnc, m)$}{%
        (x,X) := \skSign; Y := \pkEnc  \\
        r \sample \ZZ_q; \hat{R} \gets{} g^r; R \gets Y^{r} \\
        \pi \sample \pcalgostyle{P}_{\DLEQ}((g, \hat{R}), (Y, R),r) \\
        \Rx \gets \xcoord(R) \\
        \hat{s} \gets r^{-1}(H(m) + \Rx{}x)\\
        \pcreturn \hat{\sigma} := (R, \hat{R},\hat{s}, \pi) \\
    }
    \end{pchstack}
    \begin{pchstack}[center]

    \procedure{$\EncVer(\pkSign,\pkEnc, m, \hatsigma)$}{%
        X := \pkSign;  Y := \pkEnc \\
        (R, \hat{R},\hat{s}, \pi) := \hatsigma \\
        \pcalgostyle{V}_{\DLEQ}((g, \hat{R}), (Y, R)), \pi) \stackrel{?}{=} 1\\
        \Rx \gets \xcoord(R) \\
        \pcreturn \hat{R} = (g^{H(m)}X^{\Rx})^{\hat{s}^{-1}} \\
    }
    \pchspace
    \procedure{$\DecSig(\skEnc,\hatsigma)$}{
        (R, \hat{R},\hat{s}, \pi) := \hatsigma \\
        (y, Y) := \skEnc \\
        s \gets \hat{s}y^{-1} \\
        \pcreturn \sigma := (\xcoord(R),s)
    }
    \pchspace
    \procedure{$\RecKey(\pkEnc, \hatsigma)$}{
        (R, \hat{R},\hat{s}, \pi) := \hatsigma \\
        Y := \pkEnc \\
        \pcreturn \rec := (Y, \hat{s})
    }
    \pchspace
    \procedure{$\Rec(\sigma, \rec)$}{
        (\Rx,s) := \sigma; (Y, \hat{s}) := \rec \\
        \tilde{y} \gets  s^{-1}\hat{s} \\
        y := \left\{\begin{array}{l}
                \tilde{y}  \hspace{1em} \pcif g^{\tilde{y}} = Y \pclb
                -\tilde{y} \hspace{0.33em} \pcif g^{\tilde{y}} = Y^{-1} \pclb
                \bot \hspace{1em} \textbf{otherwise}
                \end{array}
             \right. \\
        \pcreturn y
    }
    \end{pchstack}
    \end{mdframed}
    \caption{The algorithms of the ECDSA one-time VES scheme. $f: \G \rightarrow \ZZ_q$ converts a an elliptic curve point to its x-coordinate mod $q$.}
    \label{fig:ecdsa-adaptor}
\end{figure}

We now prove the scheme secure by showing it meets our requirements of validity, recoverability and $\EUFCMAVES$.
In the latter case, the proof is not straightforward because we must account for a weakness in the scheme.
First we prove validity and recoverability.

\begin{lemma}
The ECDSA one-time VES is valid and unconditionally recoverable.
\end{lemma}
\begin{proof}
If $\EncVer(X,Y,m,(R, \hat{R},\hat{s}, \pi)) = 1$ then $\hat{R} = (g^{H(m)}X^{\Rx})^{\hat{s}^{-1}}$ and $\hat{R}^{y} = R$ if $\pi$ is sound. Which means $\hat{R}^{y} = R = (g^{H(m)}X^{\Rx})^{\hat{s}^{-1}y}$. Therefore $\DecSig$ produces a valid signature $\sigma := (R, \hat{s}y^{-1})$ whenever $\hatsigma$ is valid except with the negligible probability that $\pi$ is unsound. If $\sigma$ is valid then $\Rec$ will always recover $\tilde{y} \gets s^{-1}\hat{s} = (\hat{s}y^{-1})^{-1}\hat{s}$, which is either equal to $y$ or $-y$ (due to $\Verify$ only checking the x-coordinate of $R$).
\end{proof}


\subsection{Analysis of $\EUFCMAVES$ security}

The flaw in the scheme is that each valid ciphertext surreptitiously leaks the Diffie-Hellman key between the public signing key and the encryption key i.e.\ allows the receiver to compute $X^y = Y^x$.
At a high level, the problem arises because the value $s$ in the signature is computed through the product of $r^{-1}$ and $x$ and then $\pi$ reveals the product of $Y$ and $r$.
The adversary can then cancel out $r$ from the picture and is left with $Y^x$.
In detail we start with a valid ciphertext $(R, \hat{R},\hat{s}, \pi)$ on some message with signing key $X$ and encryption key $Y$.

\begin{lemma}
  \label{key-leak}
  Let $(R, \hat{R},\hat{s}, \pi)$ be a ECDSA one-time VES ciphertext.
  If $\EncVer(X,Y,m,(R, \hat{R},\hat{s}, \pi)) = 1$, for some message $m$ and encryption key $Y$, then $\CDH(X,Y) = R^{\hat{s}}Y^{-H(m)})^{\Rx^{-1}}$.
\end{lemma}
\begin{proof}
  Since $\hat{s} = r^{-1}(H(m) + \Rx{}x)$ and $R = Y^r$ we can compute $Y^x$ as follows:
  o\begin{align*}
    R^{\hat{s}} & = Y^{H(m) + \Rx} &  \\
    R^{\hat{s}}Y^{-H(m)} & =  Y^{\Rx{}x} \\
    (R^{\hat{s}}Y^{-H(m)})^{\Rx^{-1}} & = Y^{x}
  \end{align*}
\end{proof}


\newcommand{\SDH}{\textsf{SDH}}
\newcommand{\DLSDH}{\DLOG_{\OSDH}}
\newcommand{\QDLSDH}{\text{-}\DLSDH}
\newcommand{\OSDH}{\mathcal{O}_{\SDH}}


This clearly violates the spirit of $\EUFCMAVES$ which is that nothing useful should be learned from the ciphertext, other than the signature (if it can be decrypted).
To demonstrate the fidelity of our formal definitions to this idea we show there is no $\EUFCMAVES$ reduction for it

\begin{lemma}
  There is no \emph{key-preserving} reduction from $\DLOG$ to the $\EUFCMAVES$ security of the ECDSA one-time VES if the $\CDH$ problem is hard.
\end{lemma}
\begin{proof}[Proof Sketch]
  Observe that any key preserving $\DLOG$ reduction must simulate the encryption oracle $E$ without the secret key $x$.
  Note that it is not enough for this simulator to just return two random group elements $(\hat{R}, R)$ and simulate the proof $\pi$ to make them appear valid with respect to a query for an encryption under $Y$.
  We can easily catch this behaviour by querying the simulator on key $Y'$ such that we know the secret key $y'$ and checking that $\hat{R}^y = R$.
  Thus since the simulator must return valid ciphertexts and from valid ciphertexts the Diffie-Hellman key can be extracted (as shown above) the simulator must not exist if $\CDH$ problem is hard.
  If no simulator exists then no key-preserving reduction can exist.
\end{proof}


Since we believe this scheme is useful we will now attempt to salvage it by formally capturing the flaw and prove it secure in a weaker model.
The problem for the reduction above is that by providing the signature encryption oracle you are also providing what is referred to by Brown et al.\ as a \emph{static Diffie-Hellman} oracle\cite{SDHP} i.e. an oracle that will return $Y^x$ for some fixed $x$ on a query for $Y$.
Such an oracle can obviously not be simulated if the \CDH problem is hard. %TODO: It is strongly implied that it is hard.
Our solution to this conundrum is to give our reduction access to such an oracle, denoted as $\OSDH$, and to show a reduction from $\DLOG$ in this model.
Formally, our reduction is from the following weaker version of the $\DLOG$ problem.
Note we do not give $\adv$ a the group element for which is must find the discrete logarithm as in the typical $\DLOG$ experiment since it may simply query $\OSDH(g)$ to get any particular instance of a discrete logarithm problem with respect to any $g$.

\begin{definition}[Discrete log with static Diffie-Hellman oracle problem]
  An algorithm $\adv$ solves the $u\QDLSDH$ problem in a group $\G$ if the following experiment outputs 1 and $\adv$ makes $u$ or less queries to $\OSDH$.
  \begin{center}
    \fbox{
      \begin{pchstack}
      \procedure{$\DLSDH$}{
        x \sample \ZZ_q \\
        x^* \sample \adv^{\OSDH} \\
        \pcreturn  x^* \stackrel{?}{=} x
      }
      \pchspace
      \procedure{$\OSDH(Y)$}{
        \pcreturn Y^x
      }
      \end{pchstack}
    }
  \end{center}
\end{definition}

Using $u\QDLSDH$ instead of $\DLOG$ in our $\EUFCMAVES$ reduction means it no longer proves that ciphertexts contain no useful information for an adversary, but instead, that the Diffie-Hellman key $Y^x$ is the only extra thing that can be extracted from a ciphertext.
Note that the existing protocols that use ECDSA adaptor signatures sidestep this issue in their simulation based proofs by making any receiver of the signature encryption prove knowledge of the decryption key.
Obviously if they already know the decryption key $y$ then they can compute $X^y = Y^x$ without help from the signer.
We seek to prove the scheme secure without proofs of knowledge since it may not always be practical to provide such a proofs and it complicates the application of the scheme.
This security model allows us to prove the scheme secure but means we must make the following considerations whenever it is employed.

Firstly, when proving a protocol secure that uses the scheme, any time the adversary learns a ciphertext they should also be given the Diffie-Hellman key $Y^x$.
This prevents accidentally proving a scheme secure that, for example, uses a key as both an ElGamal encryption key and to create ECDSA encrypted signatures.
Practically, this issue is not as severe as it seems as none of protocols presented in Section~\ref{exisitng-protocols} are based on Diffie-Hellman problems.
Furthermore, on a Bitcoin like ledger, signing keys for transactions are usually generated randomly within the protocol and are not shared between executions, so learning a Diffie-Hellman key in one execution cannot help an adversary break the security of another.

Secondly, the scheme is only as secure as $\DLSDH$ which is a strictly easier problem than $\DLOG$.
The reason that $\DLSDH$ is easier is that the $\OSDH$ oracle actually helps the adversary break $\DLOG$ in a very subtle and unexpected way.
As shown by Brown et al.\cite{SDHP} it is possible to use $\OSDH$ to assist in computing the discrete logarithm of the static key (i.e the public signing key).
The attack works by querying $\OSDH$ with the static key itself and thereby learning non-linear functions of the static secret key \emph{in the exponent} e.g. a query for the static key $X = g^x$ returns $g^{x^2}$ and then querying with that result returns $g^{x^3}$ and so on.
In the next section, we give a fuller description of the attack and estimate how hard the $u\QDLSDH$ problem is for different values $u$.

To make our security claim we bound the adversary against the ECDSA one-time VES by the difficulty of solving $u\QDLSDH$.
As in the case of our Schnorr proof we use an existing $\EUFCMA$ reduction as our starting point.
The work of Fersch et al.\cite{ecdsa-eufcma} meticulously capture the $\EUFCMA$ security of ECDSA in a \emph{bijective random oracle model} by showing a reduction from the $\DLOG$.
Their reduction also shows the security of ECDSA relies on particular properties of the message hash function $H$.
We ignore this aspect of the reduction to focus on the relative advantage against $\DLOG$ but they can be thought of as implicit in the bounds function $B$ below.
In the ECDSA bijective random oracle model, the conversation function $f$ which converts an elliptic curve point $R$ to its modulo $q$ reduced x-coordinate $\Rx$ is modelled as a bijective random oracle and signatures are simulated by programming it.
As in the case of Schnorr, our approach is to modify the signature simulator to be a signature encryption simulator except that in this case we require the assistance of $\OSDH$.
We provide more details of the reduction prove the following theorem in Appendix~\ref{proof-ecdsa-eufcma}.

\begin{theorem}
  Let $\F$ $(\tau,Q_E,Q_\Pi,\epsilon)$-break the $\EUFCMAVES$ security of ECDSA.
  Then if $\Pi$ is modelled as a random oracle there exists an adversary
  that $(\tau_{\psidr},\epsilon_{\psidr})$-breaks the $\psi$-relative division resistance of $H$, an adversary that $(\tau_{\textsf{cr}}, \epsilon_{\textsf{cr}})$-breaks collision resistance of $H$ and inverters that $(\tau', \epsilon', Q_E + 1)$-break and $(\tau'', \epsilon'', Q_E + 1)$-break, respectively, $\DLSDH$ in $\G$ such that

  \[ \epsilon \leq \sqrt{2Q_{\Pi}^2\epsilon_{\psidr} + 2Q_{\Pi}\epsilon' } + \epsilon'' + Q_{\Pi}^2/2^L + \epsilon_{\textsf{cr}} + \frac{3QQ_E}{(q-1)/2-Q}  \]

  \hfill \break and $\tau_{\psidr} = \tau' = 2\tau + O(Q_E) + O(Q_{\Pi})$, $\tau'' = \tau + O(Q_{E})+ O(Q_{\Pi})$, $\tau_{\textsf{cr}} = \tau + O(Q_E)$ and $u = Q_E + 1$.


\end{theorem}

Simply stated, in the bijective random oracle model, one can break $u\QDLSDH$ with a $\EUFCMAVES$ forger if one can break $\DLOG$ with a $\EUFCMA$ forger.
The reduction of Fersh et al.\, and therefore our transformed reduction, loses a factor of $Q_{\Pi}$ and the square of the advantage i.e ignoring other terms $B(\epsilon', Q_S, Q_{\Pi}) \approx \sqrt{\epsilon'Q_{\Pi}}$.

\subsection{Hardness of $u\QDLSDH$ in secp256k1}

To show the scheme is practical we now give concrete estimates of how difficult $\DLSDH$ problem should is within the secp256k1 elliptic curve group used for ECDSA on Bitcoin.
The basis for our estimates is the work of Brown et al.\cite{SDHP} who show the best known algorithm for solving $u\QDLSDH$.
It does so in $O(\sqrt{q/u})$ time where $q$ is the order of the group.
The algorithm is based on the idea that if we suppose the order of $Z^*_q$ is divisible by $u$ (or some value less than $u$) and so $uv = q-1$ for some $v$, then there are subgroups of $Z^*_q$ order $u$ and $v$ since they both divide the order.
Simplifying a bit, the attack splits the problem of finding the discrete logarithm $x$ into finding the discrete logarithm of two smaller components of $x$ in the subgroups of order $u$ and $v$.
To this end the attacker queries the oracle $u$ times to compute $g^{x^{u}}$ which allows it to work in the subgroup of order $v$ where it solves the smaller $\DLOG$ problem using a modified baby-step-giant-step algorithm.
From there, they do the same for the subgroup of order $u$ and combine the results to finally produce the discrete logarithm.

In addition to $u$ oracle queries the algorithm requires $n = 2(\sqrt{u} + \sqrt{v})$ scalar multiplications in $\G$ to finally output the discrete logarithm with certainty.
Treating the oracle queries and scalar multiplications as equal, the algorithm has an optimal running time of approximately $3\sqrt[3]{q}$ where $u = \sqrt[3]{q}$.
Since the main and likely only application of this scheme will be to Bitcoin's ECDSA signature we provide concrete estimates for the amount of computation needed to run this algorithm in secp256k1 in Figure~\ref{fig:sdh_alg_time}.
As expected, the table shows the optimal value for $u$ (treating scalar multiplications between the attacker and the oracle as equal) is $~2^{84} \approx \sqrt[3]{q}$ which gives $n \approx ~2^{86}$.
Smaller and more plausible values for $u$, e.g. $u \approx 2^{24} \approx$ 16 million give $n \approx 2^{116}$ which means it is only a minor improvement on the generic algorithms for solving $\DLOG$ in elliptic curve groups which takes $\sqrt{q} \approx 2^{128}$ group operations.

\begin{figure}[h]
  \centering
  \[ q = 2^6 \times 3 \times 149 \times 631 \times 107361 793816 595537 \times 174 723607 534414 371449 \times 341 948486 974166 000522 343609 283189  \]
 \begin{tabular}{||c c c c||}
 \hline
 $\geq \log_2u $ & $\floor{\log_2u}$ & $\floor{\log_2n}$ & Factorization of $u$ \\ [0.5ex]
 \hline\hline
  * & 84 & 86 & $2 \times 149 \times 631 \times 174723607534414371449$ \\
  \hline
  80 & 80 & 88 & $2^2 \times 3 \times 631 \times 174723607534414371449$ \\
  \hline
  70 & 70 & 93 & $2^3 \times 174723607534414371449$ \\
  \hline
  60 & 60 & 98 & $2^2 \times 3 \times 107361793816595537$ \\
  \hline
  50 & 24 & 116 & $2^6 \times 3 \times 149 \times 631$ \\ [1ex]
  \hline
 \hline
\end{tabular}
\caption{The optimal number of scalar multiplications $n$ required in secp256k1 for each upper bound on the bit length of $u$ static Diffie-Hellman oracle queries to solve the $u\QDLSDH$ problem}
\label{fig:sdh_alg_time}
\end{figure}

To be confident that the ECDSA one-time VES can be used in practice we provide support for two final claims: (i) the above algorithm approximates the optimal algorithm for recovering a discrete logarithm with acess to $\OSDH$ and (ii) there is no ancillary advantage a ECDSA forger can get from $\OSDH$ other than using it to recover the discrete logarithm of the signing key.
To support the first claim we refer the reader to the work of Boneh and Boyen \cite{BBSig} that shows no generic algorithm can improve upon the complexity of the above algoirthm.
Specifically they prove a lower time bound of $\Omega(\sqrt{q/u})$  where $u < \sqrt[3]{q}$ for any generic adversary against the \emph{Strong Diffie-Hellan} problem which implies the same lower bound for $\DLSDH$.
The second claim is likely to be true since no analysis of ECDSA thus far has shown any relationship to the difficulty of forging signatures to any Diffie-Hellman type problem.
If we are accept these two propositions, then the scheme is secure for use in Bitcoin as it is today.
For perspective on the security loss, currently the only practical way to use Bitcoin script is to use \emph{pay-to-script-hash} type outputs which commit to script based spending rules with a 160-bit hash and thus only provide 80 bits of collision resistance.
Furthermore, We stress that we propose this scheme as a short term solution until Schnorr signatures are included in the Bitcoin protocol which also enables the Schnorr one-time VES.


\section{Semi-Scriptless Protocols}
\label{semi-scriptless}

\newcommand{\TxFund}{\texttt{Tx}_{\textsf{fund}}}
\newcommand{\TxRefund}{\texttt{Tx}_{\textsf{refund}}}
\newcommand{\TxRedeem}{\texttt{Tx}_{\textsf{redeem}}}
\newcommand{\TxInput}{\texttt{Tx}_{\textsf{input}}}
\newcommand{\addrRedeem}{addr_{B}}
\newcommand{\addrRefund}{addr_{A}}
\newcommand{\TxGen}{\textsf{Tx}}
\newcommand{\refundSig}{\sigma_{\textsf{refund}}}
\newcommand{\redeemSig}{\sigma_{\textsf{redeem}}}
\newcommand{\redeemEncSig}{\hatsigma_{\textsf{redeem}}}

The essential function of a smart contract on a Bitcoin-like ledger is to lock coins in an output such that they can only be spent to certain parties under certain conditions. With a smart contract language like Bitcoin script, the conditions can be expressed in the language and enforced by the ledger's transaction validation rules. In the scriptless model, we can only constrain spending through time-locks and by setting a public key, for which the spender must provide a signature. Clearly in order to stop one party from arbitrarily spending the coins the corresponding private key cannot be exclusively known to one party. Thus we require the parties have a joint ownership of the public key and cooperatively use a multi-signature protocol to sign transactions spending from the joint output. Multi-signature protocols exist for ECDSA\cite{Lindell2pECDSA,hash-proof-ecdsa}, Schnorr\cite{musig}, BLS\cite{compact-blockchains-bls} and most prominent signature schemes.

To realise most of the scriptless protocols from Section~\ref{exisitng-protocols}, the multi-signature scheme also needs to admit a two-party one-time VES encrypted signing protocol to emulate $\EncSign$ on a joint signing key. It is relatively simple to build this on top of a Schnorr multi-signature scheme, but since Schnorr signatures have not been included in the Bitcoin protocol yet, this tool is out of reach for now. Both the two-party ECDSA schemes in \cite{Lindell2pECDSA,hash-proof-ecdsa} admit a two-party $\EncSign$ protocol as described in \cite{ecdsa-scriptless-scripts}. Unfortunately, these schemes are complex and rely additional exotic computational hardness assumptions. As a result, the consensus that came out of the 2018 Lightning Developer Summit was to postpone updating the lightning specification to include ``payment points'' until Schnorr becomes viable.\cite{zmn-2pecdsa}

We present a workaround that allows protocol designers to realise many of the benefits of scriptless protocols in Bitcoin as it is today. To do so, we relax the scriptless model slightly to what we call the ``semi-scriptless'' model where protocols are allowed to use a single $\OPCHECKMULTISIG$ script opcode but no others. $\OPCHECKMULTISIG$ acts as a naive ECDSA multi-signature scheme, where the public key for the scheme is the concatenation of each party's public keys and a valid signature is the concatenation of valid signatures under each public key. When locking coins with \OPCHECKMULTISIG, a set of public keys is specified along with how many of those keys must authorize any transaction spending from it. We will only use ``2-of-2'' outputs which require two signatures on two out of two of the specified public keys.

\begin{figure}[h]
    \centering
    \begin{mdframed}
    \begin{pchstack}[center]
    \procedure{$\textsf{2p-Sign}(pk := (pk_1,pk_2), m)$}{
     \textbf{P}_1(sk_1) \< \< \textbf{P}_2(sk_2) \\
     \sigma_1 \sample \Sign(sk_1,m) \< \< \\
     \< \sendmessageright*[0.4cm]{\sigma_1} \< \\
     \< \< \Verify(pk_1,m,\sigma_1) \stackrel{?}{=} 1 \\
     \< \< \sigma_2 \sample \Sign(sk_2, m) \\
     \< \< \pcreturn \sigma := (\sigma_1,\sigma_2) \\
    }
    % \end{pchstack}
    % \begin{pchstack}[center]
    \procedure{$\textsf{2p-EncSign}(pk := (pk_1, pk_2),pk_E,m)$}{%
        \textbf{P}_1(sk_1) \< \< \textbf{P}_2(sk_2) \\
        \hatsigma_1 \sample \EncSign(sk_1,pk_E,m) \< \< \\
        \rec \gets \RecKey(pk_E, \hatsigma_1) \< \< \\
        \< \sendmessageright*[0.4cm]{\hatsigma_1} \< \\
        \< \< \EncVer(pk_1,pk_E,m,\hatsigma_1) \stackrel{?}{=} 1 \\
        \< \< \sigma_2 \sample \Sign(sk_2, m) \\
        \< \< \hatsigma := (\hatsigma_1, \sigma_2) \\
        \pcreturn \rec  \< \< \pcreturn \hatsigma
    }
    \end{pchstack}
    \end{mdframed}
    \caption{The two-party signing and encrypted signing algorithms for an 2-of-2 \OPCHECKMULTISIG output.}
    \label{opcms-2p-ves}
\end{figure}

We can transform any existing scriptless protocol into a semi-scriptless protocol by locking funds to an \OPCHECKMULTISIG 2-of-2 on two distinct public keys and using the single signer ECDSA one-time VES from Section~\ref{ecdsa-ot-ves-section}. The simple two-party signing and encrypted signing protocols are presented in Figure~\ref{opcms-2p-ves}.

The downsides of semi-scriptless protocols are readily apparent. First, the transactions are larger because they require two public keys and two signatures to spend them (in addition the overhead that comes from using script). Secondly, it is easy to distinguish \OPCHECKMULTISIG outputs from a regular payment transactions (but not from other uses of \OPCHECKMULTISIG 2-of-2).

Having said this, semi-scriptless protocols are a practical alternative to two-party ECDSA to developers who wish to attempt to realise many of the benefits of scriptless protocols prior to the Schnorr upgrade. In general, semi-scriptless enjoy better confidentiality than their script based counterparts. Although script is used, \OPCHECKMULTISIG is not particular to any protocol making it at more confidential than protocols with a unique script structure. For the following protocols we note the following particular benefits:

\begin{itemize}
    \item \textbf{Payment Channels \cite{poon2016bitcoin}:} The typical hash lock can be replaced with a discrete logarithm based lock which enables the privacy benefits from \cite{cryptoeprint:2018:472} other conjectured improvements\cite{lightning-dev-scriptless-scripts}.
    \item \textbf{Atomic swaps \cite{scriptless-atomic-swap}:} The secret that releases funds from the escrow transactions never appears on the ledger, unlike the existing hash constructions which make it easy to associate assets changing hands as the contracts on the ledgers share the same hash.
    \item \textbf{Discreet Log Contracts \cite{dryja2017discreet}:} The protocol can be completed in two transactions rather than three.
\end{itemize}

\FloatBarrier
\bibliography{bib.bib}{}
\bibliographystyle{unsrt}

\appendix
\section{ Proof for Theorem \ref{claim-ecdsa-eufcma}}

\newcommand{\Ocdh}{\mathcal{O}_{\CDH}}
\newcommand{\betarange}{\mathbb{B}}
\newcommand{\Sdleq}{\pcalgostyle{S}_{\DLEQ}}

\emph{authors note: This proof is wrong. By giving the reduction access to the
  $\Ocdh$ oracle we implicitly make the discrete logarithm problem
  easy\cite{Kushwaha16} making the reduction from the discrete logarithm
  pointless. I am in the process of fixing the proof so that it no longer
  requires such a powerful oracle and can be reduced from the discrete logarithm
  problem with a static Diffie-Hellman oracle. Technically, this is slightly easier than the
  usual discrete logarithm problem but still secure in practice.\cite{SDHP}}

\label{proof-ecdsa-eufcma}
As we have proved, the ECDSA one-time VES is not \EUFCMAVES secure if the CDH problem is hard. We now wish to show that if the CDH problem were easy then it would satisfy \EUFCMAVES i.e.\ leak no useful information to a forger. Thus we give our reduction access to an $\Ocdh$ oracle which when queried with $\Ocdh(X,Y)$ returns $Z$ such that (X,Y,Z) is a Diffie-Hellman tuple with respect to $g$. Additionally, in the reduction we simulate the NIZK $\DLEQ$ proof $\pi$ with $\Sdleq$.

As our starting point, we take the \EUFCMA reduction for ECDSA by Fersch et al.\cite{ecdsa-eufcma} to the discrete logarithm problem. In this work, the conversion function $f:\G \rightarrow \ZZ_q$ that maps a group element to its x-coordinate mod $q$ is decomposed and idealised as an oracle. The forger must query this oracle to produce a valid forgery. During the reduction, this oracle is programmable. In more detail, $f$ is decomposed as $f = \psi \circ \Pi \circ \varphi$ where:

\begin{enumerate}
    \item $\varphi$ is an invertible 2-to-1 function mapping curve points to the domain of $\Pi$ reflecting the fact that every x-coordinate belongs to two possible group elements.
    \item $\Pi$ is the bijective random oracle that the reduction is able to program.
    \item $\psi$ is an invertible function that maps the range of $\Pi$ (denoted as $\betarange$)  to $\ZZ_q$.
\end{enumerate}

We sketch the simulation of the signature encryption oracle $E$ in this model below. Note that we leave out and simplify some important details for the sake of clarity and encourage the reader to review the original proof in \cite{ecdsa-eufcma} to get a full understanding of the reduction.


\begin{center}
    \fbox{
        \begin{pchstack}
         \procedure{$\RDL(X)$}{
            Q \gets \emptyset; \Pi \gets \emptyset  \\
            \pccomment{Run the reduction from \cite{ecdsa-eufcma}} \\
            \text{...} \gets \F^{E,S}_{\EUFCMAVES}(X) \\
        }
        \procedure{Simulate $S(m)$}{
            \beta \sample \betarange \\
            \pcif (\cdot, \beta) \in \Pi: \textbf{abort} \\
            \Rx{} \gets \psi(\beta) \\
            s \sample \ZZ_q \\
            R \gets (g^{H(m)}X^{\Rx})^{s^{-1}} \\
            \alpha \gets \varphi(R) \\
            \pcif (\alpha,\cdot) \in \Pi: \textbf{abort} \\
            \Pi \gets \Pi \cup \{ (\alpha, \beta) \} \\
            Q \gets Q \cup \{m\} \\
            \pcreturn (s, \Rx)
        }
        \pchspace
        \procedure{Simulate $E(Y, m)$}{
            \beta \sample \betarange \\
            \pcif (\cdot, \beta) \in \Pi: \textbf{abort} \\
            \Rx{} \gets \psi(\beta) \\
            \hat{s} \sample \ZZ_q \\
            \hat{R} \gets (g^{H(m)}X^{\Rx})^{\hat{s}^{-1}} \\
            R \gets \Ocdh(\hat{R}, Y) \\
            \pi \gets \Sdleq((g, \hat{R}),(Y,R)) \\
            \alpha \gets \varphi(R) \\
            \pcif (\alpha,\cdot) \in \Pi: \textbf{abort} \\
            \Pi \gets \Pi \cup \{ (\alpha, \beta) \} \\
            Q \gets Q \cup \{m\} \\
            \pcreturn (R, \hat{R}, \hat{s}, \pi) \\
        }
        \end{pchstack}
    }
\end{center}

The fact that we can simulate $E$ with access to $\Ocdh$ without modifying the internals of $\RDL$ completes the proof.

% \section{Semi-Scriptless Time-Lock Protocol}

% \begin{figure}[h]
%     \centering
%     \begin{mdframed}
%     \begin{center}
%         \pseudocode{%
%         \\
%         \textbf{Alice}(Y, \addrRefund, \TxInput) \< \< \textbf{Bob}(Y,\addrRedeem) \\[0.1 \baselineskip][\hline]
%         \< \< \\[0.5\baselineskip]
%         (sk_A,pk_A) \sample \KeyGen \< \< \\
%         \< \sendmessageright*{pk_A, \TxInput, \addrRefund} \< \\
%         \< \<  (sk_B,pk_B) \sample \KeyGen \\
%         \< \< script := \texttt{OP\_CMS-2of2}(pk_A,pk_B) \\
%         \< \< \TxFund \gets \TxGen(\TxInput, script) \\
%         \< \< \TxRefund \gets \TxGen(\TxFund, \addrRefund) \\
%         \< \< \refundSig^B \sample \Sign(sk_B, \TxRefund) \\
%         \< \sendmessageleft*{pk_B, \refundSig, \addrRedeem} \< \\
%         script := \texttt{OP\_CMS-2of2}(pk_A,pk_B) \\
%         \TxFund \gets \TxGen(\TxInput, script) \< \< \\
%         \TxRefund \gets \TxGen(\TxFund, \addrRefund) \< \< \\
%         \Verify(pk_B, \TxRefund, \refundSig^{B}) \stackrel{?}{=} 1 \\
%         \refundSig^{A} \sample \Sign(sk_A, \TxRefund) \\
%         \refundSig := (\refundSig^{A},\refundSig^{B}) \\
%         \TxRedeem \gets \TxGen(\TxFund, \addrRedeem) \< \< \\
%         \redeemEncSig^A \sample \EncSign(sk_A, Y, \TxRedeem) \\
%         \rec \gets \RecKey(Y, \redeemEncSig^A) \\
%         \< \sendmessageright*{\redeemEncSig^A} \< \\
%         \< \< \TxRedeem \gets \TxGen(\TxFund, \addrRedeem) \\
%         \< \< \EncVer(pk_A, Y, \TxRedeem, \redeemEncSig^A) \stackrel{?}{=} 1 \\
%         \< \< \redeemSig^B \sample \Sign(sk_B,\TxRedeem) \\
%         \< \< \redeemEncSig := (\redeemEncSig^A, \redeemSig^B) \\
%         \pcreturn \TxFund, (\TxRefund, \refundSig), (\TxRedeem, \rec) \< \< \pcreturn \TxFund, (\TxRedeem, \redeemEncSig) \\
%     }
%     \end{center}
%     \end{mdframed}
%     \caption{Caption}
%     \label{fig:my_label}
% \end{figure}

\end{document}






% \subsection{Opacity}

% The opacity property ensures that only those who have the decryption key can extract the signature from a VES ciphertext. The definition was not orthogonal to the unforgeability of the underlying signature scheme. The adversary did not have to necessarily decrypt the ciphertext to break opacity; simply forging a totally different signature on the same message as the encrypted signature was enough to win the game. Note that the BGLS scheme has deterministic signatures so this distinction was not important. We remedy both these issues in the our new definition:

% \newcommand{\VESopacity}{\textsf{VES-Opacity}^{\adv}_{\hatSigma}}
% \begin{definition}[Opacity]
% A VES scheme $\hatSigma$ is opaque if for every PPT algorithm $\adv$, with access to a signature encryption and decryption oracle $ED$, the $\textsf{VES-Opacity}^{\adv}_{\hatSigma}$ experiment outputs 1 with at most negligible probability over the coin tosses of the oracles, $\EncGen, \KeyGen, \EncSign$ and $\adv$. Specifically, for an adversary $\adv(\pkSign,\pkEnc,m,\hatsigma)$, $ED$ responds to requests of the form $(m', \pkEnc)$ if $m' \neq m$ with $(\sigma, \hatsigma)$ such that $\EncVer(\pkSign, \pkEnc, m', \hatsigma) = 1$ and $\sigma$ is the decryption of $\hatsigma$.
% \begin{center}
%     \fbox{
%         \procedure{$\VESopacity$}{%
%             \kSign \sample \KeyGen \\
%             \kEnc \sample \EncGen \\
%             \hatsigma \sample \EncSign(\pkSign, \pkEnc, m) \\
%             \sigma \sample \adv^{E,D}(\pkSign, \pkEnc, m, \hatsigma) \\
%             \pcreturn \sigma = \DecSig(\skEnc,\hatsigma) \\
%         }
%         \pchspace
%         \procedure{Oracle $\tilde{E}_{\pkSign,\pkEnc}(m')$}{
%              \pcreturn \EncSign(\skSign,\pkEnc,m')
%         }
%         \pchspace
%         \procedure{Oracle $\tilde{D}_{\pkEnc}$}{}

%     }
% \end{center}

% \end{definition}

% It is rather straightforward to see that for a StE VES, if the encryption algorithm is at least one-way then extracting the signature without the decryption key will be hard.  First, we recall the definition of \emph{one-way under chosen plain text attack} (\textsf{OW-CPA}) secure public key encryption.

% \begin{definition}[OW-CPA]
%  An public key encryption scheme $\Pi = (\EncGen, \Enc, \Dec)$ with message space $\mathcal{M}$ is \textsf{OW-CPA} if every PPT algorithm $\adv$ the $\textsf{OW-CPA}^{\adv}_{\Pi}$ experiment outputs 1 with at most negligible probability.
%  \begin{center}
%      \fbox{
%         \procedure{$\textsf{OW-CPA}^{\adv}_{\Pi}$}{
%             \kEnc \sample \EncGen \\
%             m \sample \mathcal{M} \\
%             c \sample \Enc(\pkEnc, m) \\
%             m' \sample \adv(\pkEnc, c) \\
%             \pcreturn m' = m \\
%         }
%      }
%  \end{center}
% \end{definition}


% \begin{theorem}
% Let $\hatSigma$ be StE VES scheme with associated public key encryption scheme $\Pi$ and let $\Sigma$ be its underlying signature scheme. If $\Sigma$ is \EUFCMA and $\Pi$ is \textsf{OW-CPA} then it is opaque.
% \end{theorem}

% \begin{proof}[Proof Sketch]
% The only ways for the adversary can output $\sigma = \DecSig(\skEnc, \hatsigma)$ and break the opacity of $\hatSigma$ in the $\VESopacity$ experiment is to (i) forge the signature or (ii) decrypt the ciphertext. Since $\Sigma$ is \EUFCMA and $\hatSigma$ is StE, it is also \EUFCMAVES which means it will succeed at (i) with negligible probability and since the encryption is \textsf{OW-CPA} it will succeed at (ii) with negligible probability.
% \end{proof}
