%%% Local Variables:
%%% mode: latex
%%% TeX-master: "main"
%%% End:
\section{Preliminaries}

\subsection{Bitcoin}
We assume a fair amount of familiarity with the Bitcoin transaction structure in Section~\ref{semi-scriptless}. A Bitcoin transaction has the following elements:
\begin{itemize}
    \item A set of inputs, each of which refer to a previous output.
    \item For each input, a witness that satisfies the spending constraints specified in the output it references.
    \item For each input, a \emph{relative} time-lock which prevents the transaction from being included in the ledger until enough time has passed since the output it references was included in the ledger.
    \item A set of new outputs, each with a value and spending constraint.
    \item An \emph{absolute} time-lock, which prevents the transaction from being included in the ledger until a specific time.
\end{itemize}
Whenever a transaction is included in the ledger, its outputs are considered ``spent'' and their value is transferred to the new outputs created by the transaction (and a small fee to the miner who included them).

When we refer to a ``layer-2'' protocol we mean any protocol that is composed of messages sent between the participants and transactions sent to the ledger. A scriptless protocol means any protocol where the transactions only use a basic public key spending constraint which can only be satisfied by a signature on the spending transaction under that public key. It's important to note that a scriptless protocol can still use the time-lock constraints (and most do).

\subsection{Notation}

We analyse security asymptotically with our security parameter as $k$.
We assume that the algorithms for each scheme have been generated by some $\textsf{Setup}(1^k)$ algorithm which generates the a concrete instance of the scheme according to $k$.
By $\negl[k]$ we denote any negligible function of $k$ i.e.  $\negl[k] < 1/p(k)$  for all positive polynomials $p(k)$ and all sufficiently large values of $k$.
We say a probability is ``negligible'' if it can be expressed as $\negl[k]$ or overwhelming if it can be expressed as $1 - \negl[k]$.
A polynomial time adversary runs in time that can be upper bounded by some polynomial of $k$. A \emph{probabilistic} algorithm is also implicitly given a long string of random bits in addition to its other arguments. We denote invoking a probabilistic algorithm with $\sample$. If an algorithm is probabilistic and polynomial time we say it is a \emph{PPT} algorithm.

\subsection{The Discrete Logarithm Problem}

The concrete signature schemes we deal with (Schnorr and ECDSA) are based on the discrete logarithm problem (\DLOG). Written multiplicatively, the \DLOG problem is to find $x \in \ZZ_q$ given $(X,g)$ such that $g^x = X$, in a group $\G$ of prime order by $q$. We denote the fixed generator of a $\DLOG$ based scheme by $g$. In reality $(\G,q,g)$ are fixed by the ledger i.e.\ the Secp256k1 elliptic curve group for Bitcoin, but we reason about them as if the they were generated by the $\textsf{Setup}(1^k)$ algorithm relative to $k$. Note that when we describe schemes based on the \DLOG problem, all scalar operations are implicitly done modulo $q$.


\subsection{Signature Schemes}

\begin{defintion}[Signature Scheme]
  \label{signature_scheme}

  A signature scheme $\Sigma$ is made up of three algorithms $(\SIGNALG)$:

  \begin{itemize}
  \item $\KeyGen \rightsample (\skSign, pkSign)$: A signing key-pair generation algorithm, which randomly generates a secret signing key $\skSign$ and a public verification key $\pkSign$.
  \item $\Sign(\skSign, m) \rightarrow \sigma$: A possibly probabilistic signing algorithm that when given a message and the secret signing key $\skSign$ returns a signature $\sigma$.
  \item $\Verify(\pkSign, m, \sigma) \rightarrow \bin$: A deterministic signature verification algorithm that outputs 1 only if the signature $\sigma$ is valid against the public signature verification key $\pkSign$.
  \end{itemize}

\end{defintion}

\begin{defintion}[\EUFCMA] {
    A signature scheme $(\epsilon \tau, Q_s)$-\EUFCMA secure for all adversaries $\adv$ making $\$
}
